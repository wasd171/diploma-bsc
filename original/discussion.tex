\section{Обсуждение}

При оптимизации расстояния между волноводом и нанодиском для случая одиночного резонатора оптимальным показателем оказалось -0.01 мкм, то есть диск слегка вдвинут вглубь волновода. Можно предположить, что в такой конфигурации достигается оптимальный баланс между эффективным проникновением света из волновода в нанодиск и изолированностью резонатора, требуемой для установления в нём резонансных колебаний.

Исследуя отклик массива резонансных наночастиц, связанных с волноводом, был обнаружен ''эффект насыщения``~--- после некоторого пограничного значения метрики резонанса перестают улучшаться. Таким образом вероятно показатели, полученные в данной работе для системы из 13 нанодисков, являются предельными и не могут быть значительно улучшены просто путём увеличения количества связанных резонаторов. Возможно однако, что повторив отдельно для каждого диска процедуру оптимизации расстояния между диском и волноводом, получится добиться более высоких показателей.

Интересно сравнить полученную структуру с функциональными аналогами. Так как предполагается использование в фотонных микросхемах, то требуется выбрать метрику, учитывающую не только ''резкость`` резонанса, но и размер структуры. В качестве такой метрики предлагается взять $\brround{dT/d\lambda}_{max}/S$. Тогда чем выше данный показатель, тем более удачна оцениваемая структура. С другой стороны, время отклика структуры $\sim 1/\Delta \omega$ должно быть мало, что также крайне важно в свете потенциального использования в логических схемах. Результаты сравнения представлены в таблице \ref{tbl:functional_compare}.

\begin{table}[H]
	\centering
	\begin{tabular}{|c|c|c|}
		\hline
		Сравниваемая структура & $\brround{dT/d\lambda}_{max}/S$, мкм$^{-3}$ & $1/\Delta \omega$, с \\
		\hline
		Одиночный нанодиск & $19.7$ & $7.8 \cdot 10^{-14}$\\
		\hline
		Массив нанодисков & $19.1$ & $2.8 \cdot 10^{-13}$\\
		\hline
		Кольцевой резонатор из $Si$ \cite{Vilson2004} & $1.3 \cdot 10^3$ & $1.5 \cdot 10^{-11}$\\
		\hline
		Кольцевой резонатор из $Si_3 N_4$ \cite{Gondarenko2009} & $3 \cdot 10^3$ & $1.5 \cdot 10^{-8}$\\
		\hline
		Дисковый резонатор из $Si$ \cite{Soltani2007} & $6.7 \cdot 10^2$ & $1.0 \cdot 10^{-8}$\\
		\hline
	\end{tabular}
	\caption{Сравнение с функциональными аналогами}
	\label{tbl:functional_compare}
\end{table}

По результатам сравнения видно, что несмотря на разницу в 2--3 порядка по основной метрике $\brround{dT/d\lambda}_{max}/S$ не в пользу массива нанодисков, его ожидаемая высокая скорость отклика по прежнему делает его использование в логических фотонных схемах довольно интересным. По формальным показателям система из одного нанодиска, связанного с волноводом, превосходит конфигурацию из 13 нанодисков, однако скромная величина модуляции (5.6\%) не позволяет применять его в реальных интегральных фотонных схемах.