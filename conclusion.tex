\chapter{Заключение}

В результате выполнения бакалаврской работы были решены следующие задачи:

\begin{enumerate}[label=\arabic*)]
	\item Численно определены оптимальные параметры системы из одного нанодиска, связанного с волноводом, для эффективного возбуждения магнитного резонанса.
	\item Последовательно численно определены оптимальные параметры системы из нескольких связанных с волноводом нанодисков в зависимости от их количества.
	\item Рассмотрены как системы с односторонним, так и с двухсторонним расположением дисков относительно волновода.
	\item Обнаружен ''эффект насыщения`` резонанса от количества резонаторов и определены предельные характеристики, достижимые на подобной конфигурации.
	\item Введена метрика и проведено сравнение с существующими решениями, близкими по функциональности. 
\end{enumerate}

Ввиду крайне большого количества степеней свободы (по 3 для каждого из 13 нанодисков), задачу создания эффективного оптического переключателя на основе массива нанодисков не представляется возможным решить банальным перебором. Использованный в данной работе метод последовательных приближений является довольно простым, однако даже с его помощью возможно показать основные качественные характеристики поведения системы. В перспективе, используя модифицированные для массива Ми-резонансных нанодисков алгоритмы и методы, упомянутые в части \ref{sec:numerical_methods}, станет возможным реализация сверхбыстрого и компактного оптического переключателя в интегральном исполнении.