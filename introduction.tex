\chapter{Введение}

С каждым годом развитие технологических возможностей человечества позволяет производить искусственные структуры с характерными размерами всё меньших порядков. В настоящее время реальностью является управление формой и размерами материалов на наномасштабах, что позволяет создавать композитные материалы с заданными свойствами. Это обуславливает необходимость исследования параметров и области применения подобных метаматериалов. Особо интересным классом метаматериалов являются структуры с оптическим магнетизмом, то есть обладающие резонансом с ненулевым магнитным дипольным моментом на оптических частотах, что невозможно в обычных веществах, встречающихся в природе в естественном виде. Известна возможность возбуждения магнитного дипольного резонанса как для металлических метаматериалов, так и для чисто диэлектрических наноструктур.

При кажущейся простоте металлических решений они обладают существенным недостатком в виде больших омических потерь. Исследование нелинейно-оптических свойств требует высокой интенсивности излучения, порой достигающей порога оптической прочности материалов, что делает подобные решения неприменимыми. Неметаллические наноструктуры выгодно выделяются отсутствием подобного недостатка, однако пока что слабо изучены.

В данной дипломной работе численно определены оптимальные параметры для возбуждения магнитного резонанса в одиночном кремниевом нанодиске, связанном с диэлектрическим волноводом, и оптимизирована конфигурация системы из 20 нанодисков для потенциального создания оптического переключателя (?).