\section{Метаматериалы}

Метаматериалы являются искусственно созданными материалами с заданными свойствами. Взаимодействие метаматериалов с электромагнитными или звуковыми \cite{Zhang2009} волнами в первую очередь определяется формой и размерами отдельных и, как правило периодических, элементов, нежели свойствами веществ, из которых они изготовлены. При создании оптических метаматериалов используются микро- и наноструктуры, обладающие характерными размерами сравнимыми с длиной волны взаимодействующего электромагнитного излучения, что приводит к принципиально новым оптическим свойствам и эффектам.

Одними из ярких свойств метаматериалов является их потенциальная возможность реализации отрицательного показателя преломления и так называемого оптического магнетизма. Создав материал с изотропными диэлектрической проницаемостью $\varepsilon = \varepsilon' + i\varepsilon''$ и магнитной проницаемостью $\mu = \mu' + i\mu''$, удовлетворяющими отношению $\varepsilon'\left|\mu\right| + \mu'\left|\varepsilon\right| < 0$, возможно получить вещество с отрицательной действительной частью показателя преломления $n = n' + in'' = -\sqrt{\varepsilon\mu}$. 

Как правило, все оптические среды обладают положительными проницаемостями $\varepsilon$ и $\mu$. Известно однако, что диэлектрическая проницаемость металлов в оптическом диапазоне может принимать отрицательные значения. К сожалению, в природе не наблюдается материалов, которым одновременно присущи $\varepsilon < 0$ и $\mu < 0$.

Основоположником теории метаматериалов, выдвинувшим в своей статье \cite{Veselago1967} идею о принципиальной возможности существования материалов с отрицательной действительной компонентой показателя преломления принято считать советского физика В.\,Г.~Веселаго, хотя следует признать, что подобные среды значительно раньше обсуждались и в работах других учёных, например в работе \cite{Sivoukhin1957} Д.\,В.~Сивухина. Подтверждение существования подобных структур было получено Джоном Пендри (англ. John B. Pendry) из Имперского колледжа в Лондоне и Дэвидом Смитом (англ. David R. Smith) из Калифорнийского университета в Сан-Диего. В своей статье \cite{Pendry2000} они показали возможность применения метаматериалов с отрицательным показателем преломления для создания суперлинз~--- структур, позволяющих получать изображения с разрешением, превосходящим дифракционный предел. В 2004 году была экспериментально продемонстрирована \cite{Grbic2004} первая суперлинза, обладавшая разрешением в три раза лучше дифракционного предела в микроволновом диапазоне.