\section{Будущее численных методов в нанофотонике}
\label{sec:numerical_methods}

В настоящее время численные методы являются незаменимым инструментом для исследования наноструктур, позволяя оценивать поведение широкого круга систем без необходимости совершать реальные измерения, что существенно экономит время. К сожалению, несмотря на то, что  так называемый {\it design-by-specification}~--- автоматический подбор параметров структуры по требуемым характеристикам устройства, уже давно является реальностью в электронике, в фотонике дизайн структур по-прежнему осуществляется преимущественно вручную. Как правило, в процессе оптимизации структур варьируются параметры нескольких объектов, спроектированных человеком. Таким образом количество степеней свободы редко превышает 5.

С другой стороны, технология инверсивного дизайна на площади 1500 нм $\times$ 1500 нм и с размером ''пикселя`` 100 нм $\times$ 100 нм уже даёт невероятное ($2^{225}$) количество вариаций дизайна, что позволит создавать миниатюрные и эффективные оптические устройства.

Ранее алгоритмы проектирования, использующие множество параметров, были использованы для оптимизации дизайна отдельных устройств. Их спектр включает в себя генетические алгоритмы \cite{Gondarenko2008}, level-set methods \cite{Kao2005}, оптимизацию геометрических параметров \cite{Seliger2006} и топологических методов \cite{Elesin2012}, обычно применяемых в других областях. К сожалению, все эти методы либо работают лишь для конкретных структур, либо требуют частичного подбора параметров человеком. 

Однако в 2015 году в работе \cite{Piggott2015} был представлен полностью независимый от человека алгоритм, способный самостоятельно конструировать оптические элементы по заданным характеристикам. В этом алгоритме используется 2 метода: метод ''приоритетной цели`` (англ. {\it objective first}) и метод ''наискорейшего спуска`` (англ. {\it steepest descent}). В первом методе ограничиваются электрические поля для удовлетворения заданным характеристикам устройства, при этом допускаются нарушения уравнений Максвелла. Затем нарушения физических принципов минимизируются при помощи оптимизирующего алгоритма ADMM (Alternating Directions Method of Multipliers). В методе ''наискорейшего спуска`` электрические поля ограничиваются для удовлетворения уравнениям Максвелла и вводится функция метрики производительности, основанная на нарушениях заданных параметров эффективности устройства. Затем решая сопряжённую электромагнитную задачу и используя оптимизацию наискорейшего градиентного спуска находится локальный градиент метрики производительности. Получающиеся элементы обладают высокой эффективностью и малыми габаритами, что делает их весьма привлекательными для решения задач фотоники.