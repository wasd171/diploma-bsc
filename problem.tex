\chapter{Постановка задачи}

Как обсуждалось выше, в кремниевых нанодисках возможно возбуждение резонансов Ми. Обзор литературы также показал, что в качестве резонаторов, связанных с волноводами, использовались лишь более габаритные структуры, которые не вполне удобны для нужд фотонных микросхем. Учитывая эти факты, была поставлена цель исследования влияния резонанса, достижимого при минимальных размерах резонатора, то есть первой магнитной моды Ми, на спектр пропускания связанного с резонатором волновода. Для достижения этой цели были поставлены и решены следующие задачи:

\begin{itemize}
	\item Численно определены оптимальные параметры системы из одного нанодиска и волновода для эффективного возбуждения магнитного резонанса.
	\item Последовательно численно определены оптимальные параметры системы из нескольких связанных с волноводом нанодисков в зависимости от их количества. При этом исследовались различные конфигурации расположения дисков относительно волновода (с одной и с двух сторон).
	\item Введена метрика и проведено сравнение с существующими решениями, близкими по функциональности.
\end{itemize}